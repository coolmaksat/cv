% Adapted from:
% URL: http://nitens.org/taraborelli/cvtex

%!TEX TS-program = xelatex
%!TEX encoding = UTF-8 Unicode

\documentclass[11pt,a4paper]{article}
\usepackage{fontspec} 
\usepackage{booktabs}
\usepackage[style=numeric, doi=false, isbn=false, url=false,
natbib=true, sorting=ydnt, maxbibnames=99,
backend=biber, giveninits=true]{biblatex}

\DeclareNameAlias{sortname}{family-given}
\DeclareNameAlias{default}{family-given}

\usepackage{soul}
\usepackage{xpatch}% or use http://tex.stackexchange.com/a/40705
\usepackage[normalem]{ulem}


\makeatletter
\newbibmacro*{name:bold}[2]{%
  \edef\blx@tmp@name{\expandonce#1, \expandonce#2}%
%  \def\do##1{\ifdefstring{\blx@tmp@name}{##1}{\bfseries\listbreak}{}}%
  \def\do##1{\ifdefstring{\blx@tmp@name}{##1}{\bfseries\listbreak}{}}%
  \dolistloop{\boldnames}}
\newcommand*{\boldnames}{}
\makeatother

\xpretobibmacro{name:family}{\begingroup\usebibmacro{name:bold}{#1}{#2}}{}{}
\xpretobibmacro{name:given-family}{\begingroup\usebibmacro{name:bold}{#1}{#2}}{}{}
\xpretobibmacro{name:family-given}{\begingroup\usebibmacro{name:bold}{#1}{#2}}{}{}
\xpretobibmacro{name:delim}{\begingroup\normalfont}{}{}

\xapptobibmacro{name:family}{\endgroup}{}{}
\xapptobibmacro{name:given-family}{\endgroup}{}{}
\xapptobibmacro{name:family-given}{\endgroup}{}{}
\xapptobibmacro{name:delim}{\endgroup}{}{}

%\renewcommand{\bfseries}{\underline}

% DOCUMENT LAYOUT
\usepackage{geometry} 
\geometry{a4paper, textwidth=5.5in, textheight=8.5in, marginparsep=7pt, marginparwidth=.6in}
\setlength\parindent{0in}

% FONTS
\usepackage{color}
% \usepackage{xunicode}
% \usepackage{xltxtra}
% \defaultfontfeatures{Mapping=tex-text}
% \setromanfont [Ligatures={Common}, Numbers={OldStyle}, Variant=01]{Linux Libertine O}
% \setmonofont[Scale=0.8]{Monaco}
\usepackage{libertine}


% ---- MARGIN YEARS
\usepackage{marginnote}
%\newcommand{\amper{}}{\chardef\amper="E0BD }
\newcommand{\years}[1]{\marginnote{\small #1}}
\renewcommand*{\raggedleftmarginnote}{}
\setlength{\marginparsep}{25pt}
\reversemarginpar

% HEADINGS

\usepackage{sectsty} 
\usepackage[normalem]{ulem}
\sectionfont{\mdseries\upshape\Large}
\subsectionfont{\mdseries\scshape\Large} 
\subsubsectionfont{\mdseries\upshape\large} 


% PDF SETUP
% ---- FILL IN HERE THE DOC TITLE AND AUTHOR
\usepackage[bookmarks, colorlinks, breaklinks, 
% ---- FILL IN HERE THE TITLE AND AUTHOR
	pdftitle={Maxat Kulmanov - CV},
	pdfauthor={Maxat Kulmanov},
	pdfproducer={https://coolmaksat.github.io}
]{hyperref}  
\hypersetup{linkcolor=blue,citecolor=blue,filecolor=black,urlcolor=blue} 

% DOCUMENT
\begin{document}
\centerline{{\LARGE Maxat Kulmanov}}
\vspace{1cm}
King Abdullah University of Science and Technology\\
Computer, Electrical and Mathematical Sciences \& Engineering
Division\\
Computational Bioscience Research Center\\
4700 King Abdullah University of Science and Technology\\
Thuwal 23955-6900\\
Kingdom of Saudi Arabia\\[.2cm]
%Phone: \texttt{+966 2 808 1643}\\
%Fax: \texttt{609-924-8399}\\
email: \href{mailto:maxat.kulmanov@kaust.edu.sa}{maxat.kulmanov@kaust.edu.sa}\\
%\textsc{url}: \href{http://borg.kaust.edu.sa/}{http://borg.kaust.edu.sa/}\\ 

% \vspace{3mm}
% \noindent
% Day of birth:  May 30, 1980\\
% Place of birth: Leipzig, Germany\\
% Marital status: Single\\
% Nationality: German

%\hrule

\section*{Education}
\noindent
\years{2015-2020}King Abdullah University of Science and Technology,
Thuwal, Saudi Arabia.
  \begin{itemize}
  \item PhD in Computer Science
  % \item Member of graduate schools {\em Knowledge Representation} and
  %   {\em Leipzig School of Human Origins}
  \item Dissertation title: {\em Predicting Gene Functions and Phenotypes by combining
Deep Learning and Ontologies}
 \item Advisor: Prof. Dr. Robert Hoehndorf
  \end{itemize}
\years{2009-2010}Kazakh-British Technical University, Almaty, Kazakhstan.
\begin{itemize}
\item M.Sc. in Engineering and Technology.
\end{itemize}
\years{2005-2009}Kazakh-British Technical University, Almaty, Kazakhstan.
\begin{itemize}
\item B.Sc. in Information Systems.
\end{itemize}

% \begin{itemize}
% \item Minor subject: Logics and Philosophy of Science
% \end{itemize}
%\hrule

\section*{Professional experience}
% \noindent
\years{2021--now}{\em Research Scientist} in Computer Science,
Computer, Electrical, and Mathematical Sciences \& Engineering
Division, King Abdullah University of Science and Technology.\vspace{.3cm}\\
\years{2020--2021}{\em Postdoctoral Researcher} in Computer Science,
Computer, Electrical, and Mathematical Sciences \& Engineering
Division, King Abdullah University of Science and Technology.\vspace{.3cm}\\
\years{2014-2015}{\em Lead Software Developer},
Education Systems Production, Almaty, Kazakhstan.\vspace{.3cm}\\
\years{2012-2014}{\em Software Developer},
Executive Consulting, Almaty, Kazakhstan.\vspace{.3cm}\\
\years{2011-2012}{\em Visiting Researcher} in Software Engineering Department,
University of Oldenburg, Germany.\vspace{.3cm}\\
\years{2009-2011}{\em Lecturer}, Kazakh-British Technical University.\vspace{.3cm}\\
\years{2007-2009}{\em Software Developer}, Ferrum Logic, Almaty, Kazakhstan.\vspace{.3cm}\\

\section*{Researcher identifiers}
\begin{itemize}
\item ORCID: 0000-0003-1710-1820
\item ResearchID: AAG-5628-2021
\item SCOPUS: 57193354710
\end{itemize}

\section*{Awards and fellowships}
\noindent
\years{2019}{\em Travel Fellowship} NBDC/DBCLS 2019 BioHackathon.\vspace{.3cm}\\
\years{2018}{\em Travel Fellowship} NBDC/DBCLS 2018 BioHackathon.\vspace{.3cm}\\
\years{2015-2020}{\em PhD Fellowship}, King Abdullah
University of Science and Technology, Thuwal, Saudi Arabia. \vspace{.3cm}\\
\years{2011-2012}{\em Erasmus Mundus Scholarship}, University of
Oldenburg, Oldenburg, Germany. \vspace{.3cm}\\
\years{2009-2010}{\em M.Sc Fellowship}, Kazakh-British Technical
University, Almaty, Kazakhstan. \vspace{.3cm}\\

\section*{Professional affiliations}
\years{2016--now} International Society of Computational Biology
(ISCB)

\section*{Publications}

\forcsvlist{\listadd\boldnames}{
  {Kulmanov, M\bibinitperiod},
  {Kulmanov, M.},
}

Since I joined KAUST I published 16 papers in peer-reviewed journal
articles and 2 conference articles of which 10 were first-authored by
me.

%\section*{Refereed publications}

%\subsection*{Journal articles}

%\addbibresource{article-kaust}
\begin{refsection}[articles]
\nocite{*}
\end{refsection}
\begin{refsection}[articles-conf]
\nocite{*}
\end{refsection}

\defbibheading{subbibliography}[\refname]{\subsubsection*{#1}}
\subsection*{Peer-reviewed journal articles}
\printbibliography[section=1, heading=subbibliography]

\subsection*{Conference articles}

\printbibliography[section=2, heading=subbibliography]

\subsubsection*{Tutorials}

\years{2019}{\em Semantic similarity and machine learning with
  ontologies}. Joint Ontology Workshops (JOWO) (with Robert Hoehndorf).\vspace{.3cm}\\

\subsubsection*{Invited talks}
\years{2020}{\em DeepPheno: Predicting single gene loss-of-function
  phenotypes using an ontology-aware hierarchical
  classifier}. Function SIG Meeting, ISMB/ECCB, Virtual Conference.
Digital Health Conference, Thuwal, Saudi Arabia.\vspace{.3cm}\\
\years{2020}{\em DeepPheno: Predicting single gene knockout phenotypes.}
Digital Health Conference, Thuwal, Saudi Arabia.\vspace{.3cm}\\
\years{2019}{\em DeepGOPlus: Improving protein function prediction from sequence.}
BioHackathon, Fukuoka, Japan.\vspace{.3cm}\\
\years{2019}{\em Vec2SPARQL: integrating SPARQL queries and knowledge
  graph embeddings.} Artificial Intelligence in Medicine Conference, Thuwal,
Saudi Arabia.\vspace{.3cm}\\
\years{2018}{\em Vec2SPARQL: integrating SPARQL queries and knowledge
  graph embeddings.}
BioHackathon, Matsue, Shimane, Japan.\vspace{.3cm}\\
\years{2018}{\em Ontology-based validation and identification of
  regulatory phenotypes}. ECCB, Athens, Greece.
\years{2017}{\em DeepGO: Predicting Protein Functions from Sequence
  and Interactions Using a Deep Ontology-aware Classifier.}
Function SIG Meeting ISMB/ECCB, Prague, Czech Republic.\vspace{.3cm}\\
\years{2017}{\em DeepGO: Predicting Protein Functions from Sequence
  and Interactions Using a Deep Ontology-aware Classifier.}
Big Data Analyses in Evolutionary Biology, Thuwal,
Saudi Arabia.\vspace{.3cm}\\
\years{2016}{\em Evaluating the effect of annotation size on measures of semantic similarity.}
Bio-Ontologies SIG Meeting, ISMB, Orlando, USA.\vspace{.3cm}\\

% \section*{Research funds}

% \years{2019-2021}{\em CompleX: Variant Prioritization in Complex Disease}
% \begin{itemize}
% \item Funding body: KAUST (Competitive Research Grant)
% \item PI: Robert Hoehndorf
% \item Co-Investigators: Paul N Schofield, Georgios V Gkoutos
% \item Amount: 399,998 USD (240,000 USD to Robert Hoehndorf)
% \end{itemize}

% \years{2019}{\em Whole genome sequencing of rare disease patients.}
% \begin{itemize}
% \item Funding body: KAUST (OSR Director's Award, Digital Health Initiative)
% \item PI: Robert Hoehndorf
% \item Total amount: 138,600 USD (138,600 USD to Robert Hoehndorf)
% \end{itemize}

% \years{2018-2020}{\em Sequencing and computational analysis of MRSA samples.}
% \begin{itemize}
% \item Funding body: KACST
% \item PI: Robert Hoehndorf, Mohammed Al Fageeh
% \item Co-Investigators: Takashi Gojobori, Vladimir Bajic
% \item Total amount: 362,159 USD (362,159 USD to Robert Hoehndorf)
% \end{itemize}

% \years{2018-2019}{\em Improvement of genetic variant prioritization technology.}
% \begin{itemize}
% \item Funding body: KAUST (Center Partnership Fund)
% \item PI: Robert Hoehndorf
% \item Co-Investigators: Paul Schofield, Georgios Gkoutos, Vladimir Bajic
% \item Total amount: 129,715 USD (9,500 USD to Robert Hoehndorf)
% \end{itemize}

% \years{2018-2020}{\em Bio2Vec: Smart analytics infrastructure for the
%   life sciences.}
% \begin{itemize}
% \item Funding body: KAUST (Competitive Research Grant)
% \item PI: Robert Hoehndorf
% \item Co-Investigators: Xin Gao, Michel Dumontier, Jens Lehmann
% \item Total amount: 399,986 USD (113,250 USD to Robert Hoehndorf)
% \end{itemize}

% \years{2018-2020}{\em The Whale Shark 100: Applying Population
%   Genomics to Understand Mysteries of the World’s Largest Fish.}
% \begin{itemize}
% \item Funding body: KAUST (Competitive Research Grant)
% \item PI: Takashi Gojobori
% \item Co-Investigators: Michael Berumen, Robert Hoehndorf
% \item Amount: 389,713 USD (105,838 USD to Robert Hoehndorf)
% \end{itemize}

% \years{2016-2018}{\em Data integration and ontologies for microbial
%   cell factories.}
% \begin{itemize}
% \item Funding body: KAUST (Center Competitive Funding)
% \item PI: Vladimir Bajic
% \item Role: WP leader
% \item Amount: 4,786,036 USD (115,691 USD to Robert Hoehndorf)
% \end{itemize}

% \section*{Invited talks}

% \subsection*{Keynote and distinguished lectures}

% \subsection*{Departmental talks}
% \years{2018}{\em Learning from Semantic Biological Data.} Research
% Seminar, Warwick University, UK.\vspace{.3cm}\\
% \years{2018}{\em Learning from Semantic Biological Data.} Research
% Seminar, Swansea University, UK.\vspace{.3cm}\\
% \years{2018}{\em Learning from Semantic Biological Data.} Research
% Seminar, Aberystwyth University, UK.\vspace{.3cm}\\
% \years{2018}{\em Learning from Semantic Biological Data.} Research
% Seminar, Bangor University, UK.\vspace{.3cm}\\
% \years{2018}{\em Semantic prioritization of causative variants in
%   oligogenic disease.} Research Seminar, Leiden University Hospital,
% Holland.\vspace{.3cm}\\
% \years{2018}{\em Learning from Semantic Biological Data.} Research
% Seminar, Northeastern University, US.\vspace{.3cm}\\
% \years{2018}{\em Symbolic AI in Computational Biology.} Bioinformatics
% Research Seminar, University of Cambridge, UK.\vspace{.3cm}\\
% \years{2017}{\em Combining symbolic and statistical AI methods for biomedical data
% analysis.} Research seminar, IIIS, Tsinghua University.\vspace{.3cm}\\
% \years{2017}{\em Semantic prioritization of novel causative variants.}
% Research seminar, Peking University.\vspace{.3cm}\\
% \years{2017}{\em The Semantic Web -- Bioinformatics applications.} Lecture
% in CS, Tsinghua University.\vspace{.3cm}\\
% \years{2017}{\em Symbolic AI in Computational Biology.} Special
% Research Seminar, Scripps Research Institute.\vspace{.3cm}\\
% \years{2017}{\em Symbolic AI in Computational Biology.} Biomedical
% Informatics Research Seminar, Stanford
% University.\vspace{.3cm}\\
% \years{2017}{\em Symbolic AI in Computational Biology.} Special
% Research Seminar, University of Colorado Denver.\vspace{.3cm}\\
% \years{2017}{\em Symbolic AI in Computational Biology.} Research
% Seminar, Maastricht University.\vspace{.3cm}\\
% \years{2017}{\em Ontologies in Biology.} Colloquium in Honor of
% Prof. Dr. Heinrich Herre on the Occassion of his 75th Birthday,
% University of Leipzig.\vspace{.3cm}\\
% \years{2016}{\em Ontologies of phenotypes and their applications in
%   personalized medicine}. CS
% Seminar, University of Murcia.\vspace{.3cm}\\
% \years{2016}{\em Mobilizing and integrating phenotype
%   data}. Biodiversity-Informatics Seminar, Senckenberg Institute for
% Biodiversity.\vspace{.3cm}\\
% \years{2013}{\em From ontologies to translational medicine}. CS
% Seminar, University of Rostock.\vspace{.3cm}\\
% \years{2012}{\em From ontologies to translational medicine}. Invited
% External Speaker, European Bioinformatics Institute.\vspace{.3cm}\\
% \years{2012}{\em Phenotype informatics and translational
%   research}. IMISE Kolloqium, Institute for Medical Informatics,
% Statistics and Epidemiology, University of Leipzig.\vspace{.3cm}\\
% \years{2012}{\em Ontologies for integrating and analyzing
%   phenotypes}. Department of Computer Science, University of
% Birmingham.\vspace{.3cm}\\
% \years{2011}{\em Exploring phenotype data for information about rare
%   diseases}. Department of Computer Science, University of
% Capetown.\vspace{.3cm}\\
% \years{2011}{\em Ontologies for representing, integrating and
%   analyzing phenotypes}. AIC Seminar, Stanford Research
% Institute.\vspace{.3cm}\\
% \years{2011}{\em Towards integration of biomedical ontologies and
%   systems biology}. Computational Modeling in Biology Network
% (COMBINE).\vspace{.3cm}\\
% \years{2011}{\em Integrating systems biology and biomedical
%   ontologies}. Workshop on Modelling interoperability, European
% Bioinformatics Institute.\vspace{.3cm}\\
% \years{2010}{\em Interoperability between biomedical
%   ontologies}. Knowledge Representation and Knowledge Management
% Research Group, University Mannheim.\vspace{.3cm}\\
% \years{2010}{\em The ontology of biomedical sequences}. IMISE
% Kolloqium, Institute for Medical Informatics, Statistics and
% Epidemiology, University of Leipzig.\vspace{.3cm}\\
% \years{2010}{\em Perspectives for the ontology of
%   phenotypes}. Ontology Interest Group, European Bioinformatics
% Institute.\vspace{.3cm}\\
% \years{2010}{\em An introduction to formal ontology}. Ontology
% Interest Group, European Bioinformatics Institute.\vspace{.3cm}\\
% \years{2008}{\em Towards interoperability between anatomy and
%   phenotype ontologies}. Dagstuhl seminar {\em Ontologies and Text
%   Mining for Life Sciences : Current Status and Future
%   Perspectives}.\vspace{.3cm}\\
% \years{2007}{\em Interoperability, non-monotonicity and core
%   ontologies}. Dagstuhl seminar {\em Towards Interoperability of
%   Biomedical Ontologies}.



\section*{Teaching experience}
\subsection*{Courses}
\noindent
\years{2020}{\em Co-Instructor} for {\em Algorithms in Bioinformatics},
Computer Science Program, King Abdullah University of Science and
Technology.\vspace{.3cm}\\
\years{2013}{\em Instructor} for {\em Database Management, SQL},
Information Systems Program, University of International Business.\vspace{.3cm}\\
\years{2013}{\em Instructor} for {\em Enterprise Level Applications},
Information Systems Program, Kazakh-British Technical University.\vspace{.3cm}\\
\years{2012-2013}{\em Instructor} for {\em Programming Languages},
Information Systems Program, Kazakh-British Technical University.\vspace{.3cm}\\
\years{2010}{\em Instructor} for {\em Fundamentals of Web Development},
Information Systems Program, Kazakh-British Technical University.\vspace{.3cm}\\
\years{2010}{\em Instructor} for {\em Algorithms and Data Structures},
Information Systems Program, Kazakh-British Technical University.\vspace{.3cm}\\
\years{2009-2010}{\em Instructor} for {\em Informatics},
Information Systems Program, Kazakh-British Technical University.\vspace{.3cm}\\

% \section*{Research supervised}
% \subsection*{Supervision at KAUST}

% \begin{table}[h!]
%   \centering
%   \begin{tabular}{|l|l|l|}
%     \toprule
%     \multicolumn{3}{c}{Supervision at KAUST}\\
%     \hline
%     {\bf Primary Supervision -- Masters} & {\bf Primary Supervision -- PhD} &
%                                                                   {\bf
%                                                                               PostDoc
%                                                                               supervision}\\
%     Completed: 6 & Completed: 3 & Completed: 1\\
%     In progress: 0 & In progress: 6 & In progress: 2\\
%     \bottomrule
%   \end{tabular}
% \end{table}

% \begin{itemize}
% \item Primary supervision -- Masters:
%   \begin{itemize}
%   \item Completed: 3
%   \item In progress: 3
%   \end{itemize}
% \item Primary supervision -- PhD:
%   \begin{itemize}
%   \item Completed: 1
%   \item In progress: 5
%   \end{itemize}
% \item Postdoc supervision:
%   \begin{itemize}
%   \item Completed: 1
%   \end{itemize}
% \end{itemize}

% \subsubsection*{PhD advisor}

% \noindent
% \years{2015--2019}Imane Boudellioua, Computer Science; Semantic
% Prioritization of Novel Causative Variants (Start date: 2015,
% Graduated: 2019)
% \begin{itemize}
% \item First (current) position: Assistant Professor in Computer Science, King
%   Fahd University of Petroleum and Minerals, Saudi Arabia
% \end{itemize}
% \years{2015--2019}Mona Alshahrani, Computer Science; Multi-modal
% learning on biological knowledge graphs (Start date: 2015,
% Graduated: 2019)
% \begin{itemize}
% \item First (current) position: Assistant Professor in
%   Computer Science, Jubail University College, Saudi Arabia
% \end{itemize}
% \years{2015--2020}Maxat Kulmanov, Computer Science; Prediction of
% protein functions and phenotypes (Start date: 2015, Graduated:
% 2020)\\
% \years{2018--now}Sarah Alghamdi, Computer Science; Ontology design
% patterns for biomedical data analysis (Start date: 2018, Expected
% graduation: 2022)\\
% \years{2019--now}Sara Althubaiti, Computer Science; Variant
% prioritization in cancer (Start date: 2019, Expected
% graduation: 2023)\\
% \years{2019--now}Azza Althagafi, Computer Science; Mechanistic
% understanding of complex disease through analysis of longitudinal
% health data (Start date: 2019, Expected graduation: 2023)\\
% \years{2018--now}Jun Chen, Computer Science (Start date: 2018,
% Expected graduation: 2023)\\
% \years{2019--now}Sumyyah Toonsi, Computer Science (Start date: 2019,
% Expected graduation: 2023)

% \subsubsection*{MSc advisor}
% \years{2019--now}Sakhaa Alsaedi, Computer Science (Start date:
% 2017, Graduated: 2020)\\
% \years{2018--2019}Abeer Almutairi, Computer Science; Unsupervised
% methods for named entity recognition (Start date:
% 2017, Graduated: 2019)\\
% \years{2019}Sumyyah Toonsi, Computer Science; Automatic annotation of
% protein functions through text mining (Start date: 2017, Graduated:
% 2019)
% \begin{itemize}
% \item First (current) position: PhD student at KAUST
% \end{itemize}
% \years{2017--2018}Sarah Alghamdi, Computer Science; Ontology design
% patterns for aging mouse ontologies (Start date: 2017, Graduated:
% 2018)
% \begin{itemize}
% \item First (current) position: PhD student at KAUST
% \end{itemize}
% \years{2018}Sara Althubaiti, Computer Science; Ontology-based
% identification of cancer driver genes (Start date: 2018, Graduated:
% 2018)
% \begin{itemize}
% \item First (current) position: PhD student at KAUST
% \end{itemize}
% \years{2018} Azza Althagafi, Computer Science; Simulation and
% visualization of human genomes (Start date: 2018, Graduated:
% 2018)
% \begin{itemize}
% \item First (current) position: PhD student at KAUST
% \end{itemize}

% \subsubsection*{MSc co-advisor}
% \years{2017}Omar Maddouri, Biological and Biomedical Sciences, Hamad
% Bin Khalifa University, Qatar; Deep learning on biological knowledge
% graphs\\
% \years{2018--2019}Ian Coleman, Bioinformatics, Wageningen University, The
% Netherlands; Predicting Chemical--Disease Associations with Ontological
% Embeddings and Artificial Intelligence

% \subsubsection*{Postdoc supervised}
% \years{2016-2018} Miguel Angel Rodriguez Garcia:
% \begin{itemize}
% \item Start date: 2016
% \item Field of study: Computer Science
% \item Departure date: 2018
% \item Previous institution: University of Murcia
% \item Current position: Research scientist, King Juan Carlos
%   University, Spain
% \end{itemize}


% \years{2018--2019}{\em Co-supervision of master thesis}, Computer Science,
% King Abdullah University of Science and Technology
% \begin{itemize}
% \item Topic: Prediction of chemical toxicity
% \end{itemize}
% \years{2018}{\em Supervision of master thesis}, Computer Science,
% King Abdullah University of Science and Technology
% \begin{itemize}
% \item Topic: Visualization and Simulation of Genomes for Premarital
%   Testing.
% \end{itemize}
% \years{2018}{\em Supervision of master thesis}, Computer Science,
% King Abdullah University of Science and Technology
% \begin{itemize}
% \item Topic: Ontology-based prediction of cancer driver genes.
% \end{itemize}
% \years{2017-2018}{\em Supervision of master thesis}, Computer Science,
% King Abdullah University of Science and Technology
% \begin{itemize}
% \item Topic: Ontology Design Patterns for Combining Pathology and
%   Anatomy: Application to Study Ageing and Longevity in Inbred Mouse
%   Strains
% \end{itemize}
% \years{2016--2017}{\em Co-supervision of master thesis}, Life Sciences
% Division, Hamad Bin Khalifa University, Qatar
% \begin{itemize}
% \item Topic: deep learning on biological knowledge graphs
% \end{itemize}
% \years{2014}{\em Supervision} (3) and {\em Co-supervision} (2) for
% final year {\em BSc Computer Science} projects, Department of Computer
% Science, Aberystwyth University\\
% \years{2008-2009}{\em Co-supervision of master thesis}, Department of
% Computer Science, University of Leipzig and Max Planck Institute for
% Evolutionary Anthropology, Leipzig, Germany.
% \begin{itemize}
% \item Topic: ontology-based collaborative tagging in biomedicine.
% \end{itemize}
% \years{2006-2008}{\em Co-supervision of master thesis}, Institute for
% Logics and Philosophy of Science, University of Leipzig and Max Planck
% Institute for Evolutionary Anthropology, Leipzig, Germany.
% \begin{itemize}
%   \item Topic: use of automated reasoning in a semantic wiki for life
%     science data.
% \end{itemize}
% \years{2006-2008}{\em Co-supervision of master thesis}, Department of
% Computer Science, University of Leipzig and Max Planck Institute for
% Evolutionary Anthropology, Leipzig, Germany.
% \begin{itemize}
% \item Topic: representing $n$-ary relations and roles in a semantic
%   wiki.
% \end{itemize}

% \section*{University service}

% \years{2019--2020}{\em Directed Research Evaluation Committee}, King Abdullah
% University of Science and Technology.\\
% \years{2019}{\em Advisory committee on AI strategy}, King Abdullah
% University of Science and Technology.\\
% \years{2018--now}{\em Steering committee member} of Women in Data
% Science and Technology, King Abdullah University of Science and
% Technology.

\section*{Professional service}

\noindent

% \subsection*{Editorial work}

% Since 2014, I have been handling editor for over 50 manuscripts at 5
% journals.\\
% \years{2018--2021} Member of Editorial Board: {\em PLoS ONE}\\
% \years{2017--now} Associate Editor: {\em Applied Ontology}\\
% \years{2017--now} Associate Editor: {\em BMC Bioinformatics}\\
% \years{2016--now} Member of Editorial Board: {\em Data Science}\\
% \years{2012--now} Associate Editor: {\em Journal of Biomedical
%   Semantics}\\
% \years{2012} Editor: Special Issue on Ontologies in Biomedicine and
% Life Sciences in {\em Journal of Biomedical Semantics}\\
% \years{2011} Editor: Special Issue on Ontologies in Biomedicine and
% Life Sciences in {\em Journal of Biomedical Semantics}\\
% \years{2010} Editor: Special Issue on Ontologies in Biomedicine and
% Life Sciences in {\em Journal of Biomedical Semantics}\\

% \subsection*{Reviewer for funding organizations}
% In total, I reviewed over 50 research proposal since joining KAUST and
% participated in four grant review panels.  I have reviewed research
% grant applications for
% \begin{itemize}
% \item German Federal Ministry of Education and Research (BMBF)
% \item German Research Foundation (DFG)
% \item National Research Fund Luxembourg (FNR)
% \item Dr Hadwen Trust
% \item European Commission Horizon 2020: ERA-Net for Research
%   Programmes on Rare Diseases
% \item Alzheimer's Research UK
% \end{itemize}
% and have been an invited panel member for grant programs at
% \begin{itemize}
% \item German Federal Ministry of Education and Research (BMBF)
%   \begin{itemize}
%   \item i:DSem -- Integrative Data Semantics
%   \item Computational Life Sciences
%   \item Computational Life Sciences (Deep Learning in Life Sciences)
%   \end{itemize}
% \end{itemize}


\subsection*{Reviewer for journals}
Since joining KAUST, I reviewed manuscripts for {\em Nature Machine
Intelligence}, {\em Nature Communications}, {\em Journal of
Computational Biology and Chemistry}, {\em Bioinformatics}, {\em
Applied Ontology}, {\em BMC Bioinformatics}, {\em BMC Medical Informatics},
{\em Journal of Biomedical
Semantics}, {\em PeerJ}, {\em PLoS ONE}, and {\em Nature Scientific Reports}.

\subsection*{Conference organization}

\years{2022}{\em Organizing commitee} for Bio-Ontologies 2022\\
\years{2022}{\em PC member} of the 21st European Conference on
Computational Biology (ECCB2022)\\
\years{2021}{\em Organizing commitee} for Bio-Ontologies 2021\\
\years{2020}{\em PC Member} for Bio-Ontologies 2020\\
\years{2020}{\em PC member} of the 17th European Conference on
Computational Biology (ECCB)\\
\years{2019}{\em PC Member} for Semantic Web Applications and Tools in
Health Care and Life Sciences (SWAT4HCLS) 2019\\
\years{2018}{\em PC member} of the 17th European Conference on
Computational Biology (ECCB)\\
\years{2018}{\em PC Member} for Semantic Web Applications and Tools in
Health Care and Life Sciences (SWAT4HCLS) 2018\\



% \\
% \years{2009--2013}International Organization for Ontology and its
% Applications (IAOA)
% \begin{itemize}
% \item \years{2009--2012} sub-chair of education committee for Doctoral Consortia
% \end{itemize}

%\hrule

% \section*{References}
% The following scientists have agreed to provide a reference for me:
% \begin{itemize}
% \item Dr. Paul Schofield. Department of Physiology, Development and
%   Neuroscience, University of Cambridge, Cambridge CB2 3EG, UK,\\
%   \href{mailto:pns12@hermes.cam.ac.uk}{pns12@hermes.cam.ac.uk}.
% \item Dr. Georgios V. Gkoutos. Department of Computer Science,
%   Aberystwyth University, Aberystwyth, SY23 3DB, UK,\\
%   \href{mailto:geg18@aber.ac.uk}{geg18@aber.ac.uk}.
% \item Prof. Dr. Heinrich Herre, Institute for Medical Informatics,
%   Statistics and Epidemiology, University of Leipzig. Haertelstrasse
%   16--18, 04107 Leipzig, Germany,\\
%   \href{mailto:heinrich.herre@imise.uni-leipzig.de}{heinrich.herre@imise.uni-leipzig.de}.
% \item Dr. Dietrich Rebholz-Schuhmann. European Bioinformatics
%   Institute. Wellcome Trust Genome Campus, Hinxton, Cambridge, CB10
%   1SD, UK,\\ \href{mailto:rebholz@ebi.ac.uk}{rebholz@ebi.ac.uk}.
% \item Prof. Dr. Michel Dumontier, Stanford Center for Biomedical
%   Informatics Research, Stanford University, Medical School Office
%   Building, 1265 Welch Road, Stanford, CA, 94305-5479, USA\\
%   \href{mailto:Michel\_Dumontier@carleton.ca}{michel.dumontier@stanford.edu}.
% \item Dr. Janet Kelso. Max Planck Institute for Evolutionary
%   Anthropology. Deutscher Platz 6, 04103 Leipzig,
%   Germany,\\
%   \href{mailto:kelso@eva.mpg.de}{kelso@eva.mpg.de}.
% \end{itemize}


%\vspace{1cm}
\vfill{}
%\hrulefill

\begin{center}
{\scriptsize  Last updated: \today\- •\- 
}
\end{center}

\end{document}

%%% Local Variables:
%%% mode: latex
%%% TeX-master: t
%%% End:
